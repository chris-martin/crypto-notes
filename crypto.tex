\documentclass[11pt]{article}
\date{}
\title{Notes from CS 6260 (Applied Cryptography)\\Georgia Tech, Fall 2012}
\author{Christopher Martin\\{\tt chris.martin@gatech.edu}}

\usepackage{hyperref}
\usepackage{graphicx}
\usepackage{wrapfig}
\usepackage{caption}
\usepackage{subcaption}
\usepackage{amsmath,amsfonts,amssymb,amsthm}

% bold math in headers
\makeatletter
\DeclareRobustCommand*{\bfseries}{%
  \not@math@alphabet\bfseries\mathbf
  \fontseries\bfdefault\selectfont
  \boldmath
}
\makeatother

\renewcommand{\labelitemi}{}

\newcommand{\eqdef}{\ensuremath{\equiv}}

\newcommand{\Gen}{\ensuremath{\mathsf{Gen}}}
\newcommand{\Enc}{\ensuremath{\mathsf{Enc}}}
\newcommand{\Dec}{\ensuremath{\mathsf{Dec}}}

\newcommand{\eps}{\ensuremath{\varepsilon}}

\newcommand{\bit}{\ensuremath{\{\texttt{0},\texttt{1}\}}}

\theoremstyle{remark}
\newtheorem{thm}{Theorem}

\begin{document}

\maketitle

\section{Symmetric cryptography scheme}

\begin{tabular}{r|l}
Key space & $\mathcal{K}$ \\
Message space & $\mathcal{M}$ \\
Cypher space & $\mathcal{C}$ \\
Key generator & $\Gen : \phi \rightarrow \mathcal{K}$ \\
Encryption function &
$\Enc : \{\mathcal{K} \times \mathcal{M}\} \rightarrow \mathcal{C}$ \\
Decryption function &
$\Dec : \{\mathcal{K} \times \mathcal{C}\} \rightarrow \mathcal{M}$
\end{tabular}

\section{Information theoretic security}

Information theoretic security repels even resource-unbounded attackers.
Shannon secrecy and perfect secrecy are equivalent definitions of
information theoretic security for symmetric cryptography schemes.

\paragraph{Shannon secrecy}
A scheme is Shannon-secret with respect to the distribution $D$ over $\mathcal{M}$
iff the ciphertext reveals no additional information about the message.
\[ \forall \, M \in \mathcal{M}, \, C \in \mathcal{C}: \;
  \Pr_{\substack{k \leftarrow \Gen\\m \in D}} \big[\, m = M \,\vert\, \Enc_k(m) = C \,\big]
= \Pr_{m\in D} \big[\, m = M \,\big] \]

\paragraph{Perfect secrecy}
A scheme is perfectly secret iff the distributions of ciphertexts for any
two messages are identical.
\[\forall\, M_1, M_2 \in \mathcal{M},\, C \in \mathcal{C}:\;
  \Pr_{K_1\leftarrow\Gen} \big[\, \Enc_{K_1}(M_1) = C \,\big]
= \Pr_{K_2\leftarrow\Gen} \big[\, \Enc_{K_2}(M_2) = C \,\big] \]

This model considers only a single message and ciphertext,
so although a one-time pad is perfectly secret, a ``two-time pad'' is not.

\begin{thm}
Perfect secrecy $\Rightarrow$ $|\mathcal{K}| \ge |\mathcal{M}|$.
\begin{proof}
If not, $\exists$ $2$ messages with different probabilities
of encrypting to the same cypertext.
\end{proof}
\end{thm}

\section{Pseudo-random functions}

\paragraph{Uniformly random function}
$U$ is a random variable chosen uniformly from the set of all functions
$\{0,1\}^m \rightarrow \{0,1\}^n$.

\paragraph{Pseudo-random function}
A PRF belongs to a family of functions
$F : \{0,1\}^\ell \times \{0,1\}^m \rightarrow \{0,1\}^n$.
Write $F_k(\cdot)$ to denote $F(k, \cdot)$.

\paragraph{Distinguishing advantage}
Consider an adversary $\mathcal{A}$ who knows $F$, having oracle access
to $F_k$ where $k$ was chosen uniformly at random, trying to
distinguish the oracle's responses from a random function.
The distinguishing advantange of $\mathcal{A}$ against $F$ is
\[ \textrm{Adv}_F^\textrm{prf}(\mathcal{A}) \eqdef
\Pr_{k\in\{0,1\}^\ell} \big[\, \mathcal{A}^{F_k}\;\textrm{accepts} \,\big]
- \Pr_U \big[\, \mathcal{A}^U \textrm{ accepts} \,\big] \;\textrm{.} \]

In time $O(t)$, we can brute-force $t$ keys to get
advantage $t / 2^\ell$.

\paragraph{$(t,q)$-bounded adversary}
\begin{tabular}{r|l}
$t$ & Running time \\
$q$ & Number of queries
\end{tabular}

\paragraph{$(t,q,\eps)$-secure PRF}
$F$ is $(t,q,\eps)$-secure iff
$\forall$ $(t,q)$-bounded $\mathcal{A}$,
\[ \textrm{Adv}_F^\textrm{prf}(\mathcal{A}) \le \eps \;\textrm{.} \]

\paragraph{Examples of reasonable constants}
\begin{tabular}{r|l}
$t$ & $2^{128}$ \\
$q$ & $2^{64}$ or $2^{32}$ \\
$\eps$ & $2^{-128}$
\end{tabular}

\paragraph{Existence}
The existence of secure PRFs has not been proven,
but there are some functions that have never been
broken and are widely assumed to be PRFs.

\section{Reduction}

\paragraph{Karp (many-to-one) reduction}
Reduction from $A$ to $B$ transforms
an instance of $A$ to an instance of $B$.

\paragraph{Cook (Turing) reduction}
Reduction from $A$ to $B$ solves $A$
using a subroutine that solves $B$.

\paragraph{Key recovery security}
$F$ is $(t,q,\eps)$-kr-secure
iff $\forall$ $(t,q)$-bounded $\mathcal{A}$,
\[ \textrm{Adv}_F^\textrm{kr} \eqdef
\Pr_{k \in \{0,1\}^\ell} \big[\,
  \mathcal{A}^{F_k(\cdot)}\;\textrm{outputs}\;k
\,\big] \le \eps \;\textrm{.} \]

\begin{thm}
If $F$ is a $(t,q,\eps)$-secure PRF
for $q < 2^m$, then
$F$ is $(t',q',\eps')$-kr-secure for
$t' \approx t$, $q' = q-1$, $\eps' = \epsilon + 2^{-n}$.
\begin{proof}
Cook reduction.
For any kr-adversary $\mathcal{A'}$ running in
time $t'$ and making $q'<2^m$ queries, let $\mathcal{A}$ be
the PRF adversary:
\begin{quote}
$k' \; \leftarrow \; \mathcal{A}'(\mathcal{O})$ \\
$x \; \leftarrow$ a value that $\mathcal{A}'$ did not query with \\
$y \; \leftarrow \; \mathcal{O}(x)$ \\
Accept iff $y = F_{k'}(x)$
\end{quote}
$\mathcal{A}$ runs in time $t \approx t'$ and makes $q = q'+1$ queries.
\[ \textrm{Adv}_F^\textrm{prf}(\mathcal{A}) \ge
\textrm{Adv}_F^\textrm{kr}(\mathcal{A}') - 2^{-n}
\;\textrm{.} \qedhere \]
\end{proof}
\end{thm}

\paragraph{Example PRF construction}
For PRF $F : \bit^\ell \times \bit^n \rightarrow \bit^n$,
$F'_k(x) \eqdef F_k(F_k(x)) \Vert F_k(\overline{F_k(x)})$

\begin{thm}
$F'$ is $\displaystyle (t, \frac{q}{3}, \eps + \frac{q^2}{2^n})$-secure.
\begin{proof}
Let $\mathcal{A}'$ be an attacker on $F'$.
Define $\mathcal{A}$ as:
\begin{quote}
$\mathcal{O}' \eqdef
\mathcal{O}(\mathcal{O}(x)) \Vert \mathcal{O}(\overline{\mathcal{O}(x)})$
(done with $3$ queries to $\mathcal{O}$)

Accept iff $\mathcal{A}'^{\,\mathcal{O}'}$ accepts
\end{quote}
$\mathcal{O}'(x)$ simulates $F'$ perfectly, so
$\displaystyle
\Pr_k \big[\, \mathcal{A}^{F_k} \,\big] =
\Pr_k \big[\, \mathcal{A'}^{\,F'_k} \,\big]$.

$\mathcal{O}'$ does not simulate $U$ perfectly, but it is close.
We have independence as long as all of the
$\mathcal{O}(x)$, $\overline{\mathcal{O}(x)}$ are distinct.
Using union bound, this probability $\le \frac{q^2}{2^n}$
\end{proof}
\end{thm}

\section{Pseudo-random permutations}

In a permutation family
$F : \{0,1\}^\ell \times \{0,1\}^n \rightarrow \{0,1\}^n$,
every $F_k$ is bijective.

A secure PRP is computationally indistinguishable from
a uniformly random permutation.

\paragraph{Some well-known PRPs}
\begin{tabular}{l|lll}
$\mathsf{DES}$ & $\ell = 56$ & $n = 64$ \\
$\mathsf{AES_{128}}$ & $\ell = 128$ & $n = 128$ \\
$\mathsf{AES_{192}}$ & $\ell = 192$ & $n = 128$
\end{tabular}

\paragraph{Strong PRP / block cipher}
Attackers with oracle access to both $F$ and $F^{-1}$
have small advantage. \[
\text{Adv}_F^\text{sprp} \eqdef
  \Pr_k \big[\, \mathcal{A}^{F_k,F_k^{-1}} \; \text{accepts} \,\big]
- \Pr_P \big[\, \mathcal{A}^{P,P^{-1}} \; \text{accepts} \,\big]
\le \eps
\]

\paragraph{PRF/PRP switching lemma}
If $G$ is a $(t, q, \eps)$-secure PRP (not necessarily strong),
then $F$ is a $(t, q, \eps + \frac{q^2}{2^{n+1}})$-secure PRF.

\section{Secure symmetric encryption}

Perfect secrecy is impossible where $m > \ell$, but
computational security is possible with pseudorandom objects.

\paragraph{Electronic code block (ECB)}
Suppose $F$ is a secure PRP
$\bit^\ell\times\bit^n\rightarrow\bit^n$
with $F$ and $F^{-1}$ efficiently computable.
~\\[5pt]
\begin{tabular}{r|l}
\Gen & $k \leftarrow \bit^\ell$ \\[5pt]
\Enc & $M'$ $\leftarrow$ Pad message $M$
       with \texttt{1} and some \texttt{0}s
       to a multiple of $n$. \\
     & Break $M'$ into $n$-bit blocks $m_0, m_1, \ldots$ \\
     & Apply $F_k$ to each of the $\{m\}$ \\[5pt]
\Dec & Apply $F'_k$ to each of the $\{m\}$
\end{tabular}\\\\
Repeated blocks give repeated ciphertext.
Never use ECB.

\paragraph{Security model}
Adversary, seeing all cipthertexts and
having oracle access to $\Enc_k$,
learns nothing about plaintexts
(except message length, which is unavoidable).

$SE = (\Gen, \Enc, \Dec)$ is
$(t, \sigma, \eps)$-IND-CPA secure
(``indistinguishable under chosen-plaintext attack'')
iff $\forall$ $(t, \sigma)$-bounded $\mathcal{A}$, \[
\text{Adv}_{SE}^\text{indcpa}(\mathcal{A}) \eqdef
\Pr_{k \leftarrow \Gen} \big[\, \mathcal{A}^{L_k} \; \text{accepts} \,\big]
- \Pr_{k \leftarrow \Gen} \big[\, \mathcal{A}^{R_k} \; \text{accepts} \,\big]
\;\text{,}
\]
\[ L_k(m, m') \eqdef \Enc_k(m)
\;\text{if}\; |m| = |m'|\;\text{else}\; \bot \;\text{,} \]
\[ R_k(m, m') \eqdef \Enc_k(m')
\;\text{if}\; |m| = |m'|\;\text{else}\; \bot \;\text{,} \]
$t$ is the running time, and
$\sigma$ total length of all message queries.

Equivalent definition:
$\Enc_k$ is computationally indistinguishable from
a zero-encrypting oracle $Z_k \eqdef \Enc_k(\texttt{0}^m)$.

\paragraph{Stateful counter mode (CTRS)}
Let $F$ be a PRF with $m = n$.
~\\[5pt]
\begin{tabular}{r|l}
\Gen & $k \leftarrow \bit^\ell$, $counter \leftarrow 0$ \\[5pt]
\Enc & echo $counter$ \\
     & for each message block $m$: \\
     & ~~~~echo $F_k(counter) \oplus m_i$ \\
     & ~~~~increment $counter$
\end{tabular}\\

If $F$ is a $(t,q,\eps)$-secure PRF, then
CTRS$(F)$ is $(t'\approx t, qn, 2\eps)$-IND-CPA secure.

\end{document}
