\documentclass[11pt]{article}
\date{}
\title{Notes from CS 6260 (Applied Cryptography)\\Georgia Tech, Fall 2012}
\author{Christopher Martin\\{\tt chris.martin@gatech.edu}}

\usepackage{hyperref}
\usepackage{graphicx}
\usepackage{wrapfig}
\usepackage{caption}
\usepackage{subcaption}
\usepackage{amsmath,amsfonts,amssymb,amsthm}
\usepackage{sidecap}

% bold math in headers
\makeatletter
\DeclareRobustCommand*{\bfseries}{%
  \not@math@alphabet\bfseries\mathbf
  \fontseries\bfdefault\selectfont
  \boldmath
}
\makeatother

\renewcommand{\labelitemi}{}

\newcommand{\eqdef}{\ensuremath{\equiv}}

\newcommand{\Gen}{\ensuremath{\mathsf{Gen}}}
\newcommand{\Enc}{\ensuremath{\mathsf{Enc}}}
\newcommand{\Dec}{\ensuremath{\mathsf{Dec}}}

\newcommand{\eps}{\ensuremath{\varepsilon}}

\theoremstyle{remark}
\newtheorem{thm}{Theorem}

\begin{document}

\maketitle

\section{Symmetric cryptography scheme}

\begin{tabular}{r|l}
Key space & $\mathcal{G}$ \\
Message space & $\mathcal{M}$ \\
Cypher space & $\mathcal{C}$ \\
Key generator & $\Gen : \phi \rightarrow \mathcal{G}$ \\
Encryption function &
$\Enc : \{\mathcal{G} \times \mathcal{M}\} \rightarrow \mathcal{C}$ \\
Decryption function &
$\Dec : \{\mathcal{G} \times \mathcal{C}\} \rightarrow \mathcal{M}$
\end{tabular}

\section{Information theoretic security}

Information theoretic security repels even resource-unbounded attackers.
Shannon secrecy and perfect secrecy are equivalent definitions of
information theoretic security for symmetric cryptography schemes.

\paragraph{Shannon secrecy}
A scheme is Shannon-secret with respect to the distribution $D$ over $\mathcal{M}$
iff the ciphertext reveals no additional information about the message.
\[ \forall \, M \in \mathcal{M}, \, C \in \mathcal{C}: \;
  \Pr_{\substack{k \in \Gen\\m \in D}} \big[\, m = M \,\vert\, \Enc_k(m) = C \,\big]
= \Pr_{m\in D} \big[\, m = M \,\big] \]

\paragraph{Perfect secrecy}
A scheme is perfectly secret iff the distributions of ciphertexts for any
two messages are identical.
\[\forall\, M_1, M_2 \in \mathcal{M},\, C \in \mathcal{C}:\;
  \Pr_{K_1\in\Gen} \big[\, \Enc_{K_1}(M_1) = C \,\big]
= \Pr_{K_2\in\Gen} \big[\, \Enc_{K_2}(M_2) = C \,\big] \]

This model considers only a single message and ciphertext,
so although a one-time pad is perfectly secret, a ``two-time pad'' is not.

\begin{thm}
Perfect secrecy $\Rightarrow$ $|\mathcal{K}| \ge |\mathcal{M}|$.
\begin{proof}
If not, $\exists$ $2$ messages with different probabilities
of encrypting to the same cypertext.
\end{proof}
\end{thm}

\section{Pseudo-random functions}

\paragraph{Uniformly random function}
$U$ is a random variable chosen uniformly from the set of all functions
$\{0,1\}^m \rightarrow \{0,1\}^n$.

\paragraph{Pseudo-random function}
A PRF belongs to a family of functions
$F : \{0,1\}^\ell \times \{0,1\}^m \rightarrow \{0,1\}^n$.
Write $F_k(\cdot)$ to denote $F(k, \cdot)$.

\begin{SCtable}[2.5][h]
\caption*{Dimensions of some well-known PRFs}
\begin{tabular}{l|lll}
$\mathsf{DES}$ & $\ell = 56$ & $m = 64$ & $n = 64$ \\
$\mathsf{AES_{128}}$ & $\ell = 128$ & $m = 128$ & $n = 128$ \\
$\mathsf{AES_{192}}$ & $\ell = 192$ & $m = 128$ & $n = 128$
\end{tabular}
\end{SCtable}

\paragraph{Distinguishing advantage}
Consider an adversary $\mathcal{A}$ who knows $F$, having oracle access
to $F_k$ where $k$ was chosen uniformly at random, trying to
distinguish the oracle's responses from a random function.
The distinguishing advantange of $\mathcal{A}$ against $F$ is
\[ \textrm{Adv}_F^\textrm{prf}(\mathcal{A}) \eqdef
\Pr_{k\in\{0,1\}^\ell} \big[\, \mathcal{A}^{F_k(\cdot)}\;\textrm{accepts} \,\big]
- \Pr_U \big[\, \mathcal{A}^{U(\cdot)} \textrm{ accepts} \,\big] \;\textrm{.} \]

In time $O(t)$, we can brute-force $t$ keys to get
advantage $t / 2^\ell$.

\paragraph{$(t,q)$-bounded adversary}
\begin{tabular}{r|l}
$t$ & Running time \\
$q$ & Number of queries
\end{tabular}

\paragraph{$(t,q,\eps)$-secure PRF}
$F$ is $(t,q,\eps)$-secure iff
$\forall$ $(t,q)$-bounded $\mathcal{A}$,
\[ \textrm{Adv}_F^\textrm{prf}(\mathcal{A}) \le \eps \;\textrm{.} \]

\begin{SCtable}[2.5][h]
\caption*{Examples of reasonable constants}
\begin{tabular}{r|l}
$t$ & $2^{128}$ \\
$q$ & $2^{64}$ or $2^{32}$ \\
$\eps$ & $2^{-128}$
\end{tabular}
\end{SCtable}

The existence of secure PRFs has not been proven,
but there are some functions that have never been
broken and are widely assumed to be PRFs.

\section{Reduction}

\paragraph{Karp (many-to-one) reduction}
Reduction from $A$ to $B$ transforms
an instance of $A$ to an instance of $B$.

\paragraph{Cook (Turing) reduction}
Reduction from $A$ to $B$ solves $A$
using a subroutine that solves $B$.

\paragraph{Key recovery security}
$F$ is $(t,q,\eps)$-kr-secure
iff $\forall$ $(t,q)$-bounded $\mathcal{A}$,
\[ \textrm{Adv}_F^\textrm{kr} \eqdef
\Pr_{k \in \{0,1\}^\ell} \big[\,
  \mathcal{A}^{F_k(\cdot)}\;\textrm{outputs}\;k
\,\big] \le \eps \;\textrm{.} \]

\begin{thm}
If $F$ is a $(t,q,\eps)$-secure PRF
for $q < 2^m$, then
$F$ is $(t',q',\eps')$-kr-secure for
$t' \approx t$, $q' = q-1$, $\eps' = \epsilon + 2^{-n}$.
\begin{proof}
Cook reduction.
For any kr-adversary $\mathcal{A'}$ running in
time $t'$ and making $q'<2^m$ queries, let $\mathcal{A}$ be
the PRF adversary:
\begin{quote}
$k' \; \leftarrow \; \mathcal{A}'(\mathcal{O})$ \\
$x \; \leftarrow$ a value that $\mathcal{A}'$ did not query with \\
$y \; \leftarrow \; \mathcal{O}(x)$ \\
Accept iff $y = F_{k'}(x)$
\end{quote}
$\mathcal{A}$ runs in time $t \approx t'$ and makes $q = q'+1$ queries.
\[ \textrm{Adv}_F^\textrm{prf}(\mathcal{A}) \ge
\textrm{Adv}_F^\textrm{kr}(\mathcal{A}') - 2^{-n}
\;\textrm{.} \qedhere \]
\end{proof}
\end{thm}

\end{document}
