
\documentclass[11pt]{article}
\date{}
\title{Cryptography notes\\CS 6260 (Fall 2012) and CS 7560 (Spring 2013)}
\author{Christopher Martin\\{\tt chris.martin@gatech.edu}}

\usepackage{hyperref,graphicx,wrapfig,caption,subcaption,multirow}
\usepackage{amsmath,amsfonts,amssymb,amsthm,dsfont,rotating}

% bold math in headers
\makeatletter
\DeclareRobustCommand*{\bfseries}{%
  \not@math@alphabet\bfseries\mathbf
  \fontseries\bfdefault\selectfont
  \boldmath
}
\makeatother

\renewcommand{\labelitemi}{}

\newcommand{\eqdef}{\ensuremath{\equiv}}

\newcommand{\Gen}{\ensuremath{\mathsf{Gen}}}
\newcommand{\Enc}{\ensuremath{\mathsf{Enc}}}
\newcommand{\AEnc}{\ensuremath{\mathsf{AEnc}}}
\newcommand{\EEnc}{\ensuremath{\mathsf{EEnc}}}
\newcommand{\Dec}{\ensuremath{\mathsf{Dec}}}
\newcommand{\Tag}{\ensuremath{\mathsf{Tag}}}
\newcommand{\Ver}{\ensuremath{\mathsf{Ver}}}

\newcommand{\eps}{\ensuremath{\varepsilon}}

\newcommand{\bit}{\ensuremath{\{\texttt{0},\texttt{1}\}}}

\newcommand{\ang}[1]{\ensuremath{\left\langle#1\right\rangle}}
\newcommand{\abs}[1]{{\ensuremath{\left\vert#1\right\vert}}}

\newcommand{\ZZ}{\ensuremath{\mathds{Z}}}
\newcommand{\N}{\ensuremath{\mathds{N}}}
\newcommand{\QR}{\ensuremath{\mathds{QR}}}
\newcommand{\I}{\ensuremath{\mathds{1}}}
\newcommand{\divides}{\mid}
\newcommand{\pk}{\text{pk}}
\newcommand{\sk}{\text{sk}}
\renewcommand{\AE}{\text{AE}}
\newcommand{\Adv}{\text{Adv}}
\newcommand{\indcpa}{\text{indcpa}}
\newcommand{\K}{\ensuremath{\mathcal{K}}}
\newcommand{\M}{\ensuremath{\mathcal{M}}}
\newcommand{\C}{\ensuremath{\mathcal{C}}}
\newcommand{\ppt}{\textit{ppt}}
\newcommand{\nuppt}{\textit{nuppt}}
\newcommand{\negl}{\text{negl}}
\newcommand{\poly}{\text{poly}}
\newcommand{\cat}{\ensuremath{\,\Vert\,}}
\newcommand{\AAA}{\ensuremath{\mathcal{A}}}
\newcommand{\BBB}{\ensuremath{\mathcal{B}}}
\newcommand{\OOO}{\ensuremath{\mathcal{O}}}

\theoremstyle{remark}
\newtheorem{thm}{Theorem}

\begin{document}

\maketitle

\section{Symmetric cryptography scheme}

\begin{tabular}{r|l}
Key space & \K{} \\
Message space & \M{} \\
Cypher space & \C{} \\
Key generator & $\Gen : \phi \rightarrow \K$ \\
Encryption function & $\Enc : \{ \K \times \M \} \rightarrow \C$ \\
Decryption function & $\Dec : \{ \K \times \C \} \rightarrow \M$
\end{tabular}

\section{Information theoretic security}

Information theoretic security repels even resource-unbounded attackers.
Shannon secrecy and perfect secrecy are equivalent definitions of
information theoretic security for symmetric cryptography schemes.

Information security turns out to be a stronger definition than
necessary, so generally we will instead consider computational security,
which will impose computation bounds on attackers.

\paragraph{Shannon secrecy}
A scheme is Shannon-secret iff the ciphertext reveals no additional
information about the message.

$\forall$ distribution $D$ over $M$, $\bar{m} \in \M$, $\bar{c} \in \C$:
\[ \Pr_{\substack{k \gets \Gen\\m \in D}} \big[\, m = \bar{m} \,\vert\, \Enc_k(m) = \bar{c} \,\big]
 = \Pr_{m\in D} \big[\, m = \bar{m} \,\big] \]

\paragraph{Perfect secrecy}
A scheme is perfectly secret iff the distributions of ciphertexts for any
two messages are identical.

$\forall\, m_1, m_2 \in \M,\, c \in \C$:
\[ \Pr_{k_1\gets\Gen} \big[\, \Enc_{k_1}(m_1) = c \,\big]
 = \Pr_{k_2\gets\Gen} \big[\, \Enc_{k_2}(m_2) = c \,\big] \]

This model considers only a single message and ciphertext,
so although a one-time pad is perfectly secret, a ``two-time pad'' is not.

\begin{thm}
Perfect secrecy $\Rightarrow$ $\abs{\K} \ge \abs{\M}$.
\begin{proof}
Let $D$ be uniform over $\M$, and
let $\bar{m} \in \M$, $\bar{c} \in \Enc_k(\bar{m})$, $\bar{k}\in \K$ be arbitrary.
Let $S = \{ \Dec_k(\bar{c}) : k \in \K \}$.
$\abs{S} < \abs{\M}$ because $\Dec$ must be deterministic.
\[ \Pr_{m \gets D}[\;m = \hat{m}\;] = \frac{1}{\abs{\M}} \; \forall \; \hat{m} \in \M \; \text{.} \]

If we choose $\hat{m} \in \M\backslash S$, then
\[ \Pr_{\substack{m\gets D,\\k \gets \Gen}}[\; m = \hat{m} \,\vert\, \Enc_k(m) = \bar{c}\;] = 0 \; \text{.} \]
\end{proof}
\end{thm}

\section{One-way functions}

A function $f: \bit^* \rightarrow \bit^*$ is a one-way-function if:
\begin{itemize}
\item Easy to compute: $\exists$ a \ppt{} algorithm $F$ such that $F(x) = f(x)$ $\forall$ $x$.
\item Hard to invert: $\forall$ \nuppt{} $A$, the advantage
  \[ \Pr_{x\gets \bit^n}[\;
     A(\underbrace{\;\;\;\;\;\;\texttt{1}^n\;\;\;\;\;\;}_{\text{helps bound $A$}}, f(x))
     \in \underbrace{f^{-1}(f(x))}_{\text{preimages of $f(x)$}}
  \;] = \negl(n) \]
\end{itemize}

The domain and range are given as $\bit^*$ for convenience,
but any $D_n \rightarrow R_n$ is okay if $D_n$ can be efficiently sampled.

\paragraph{Subset-sum}
$f_{\text{ss}}$: $(\ZZ_N)^n \times \bit^n \rightarrow (\ZZ_N)^n \times \ZZ_N$, $N = 2^n$

$f_{\text{ss}}(a_1, \ldots, a_n, b_1, \ldots, b_n) = (a_1, \ldots, a_n, S = \sum_{i=1}^n a_i b_i \bmod N)$

Subset-sum is NP-complete. This might be a one-way function.

\paragraph{Prime multiplication}
$f : (\text{primes}_n)^2 \rightarrow \N$

$f(x,y) = xy$ might be a one-way function.

\paragraph{Weak one-way functions}

For a weak one-way function, any adversary has advantage $\le 1-\delta$
for some $\delta = \frac{1}{\text{poly}(n)}$.
Multiplication might be a weak one-way function, because there are a lot of primes.

\paragraph{Hardness amplification}

We can use a weak one-way function to find a one-way function.
If $f$ is $\delta$-OWF, then $f'$ is strong-OWF where
\[ f'(x_1, \ldots, x_m) = f(x_1), \ldots, f(x_m) \]

Assume $\AAA'$ with non-negl advantange $\alpha(n)$ against $f'$.

\[ G_i \eqdef \left\{ x_i : \Pr_{\substack{x_j\gets\bit^n,\\j\neq i}}[\;
  \AAA'(f'(x'=(x_1 \ldots x_m))) \; \text{inverts}
\;] \ge \frac{\alpha}{2m} \right\} \]

For $m \ge \frac{2n}{\delta}$,
$\exists$ $i$ such that $\frac{\abs{G_i}}{2^n} \ge 1-\delta/2$.
Proof: Suppose not.
\begin{align*}
\Pr_{x'}[\;\AAA'(f'(x'))\;\text{inverts}\;]
&\le \Pr_{x'}[\;\AAA'\;\text{inverts}\;\wedge\;\text{every}\;x_i\in G\;] \\
 &\hspace{3em}+ \sum_{i=1}^m \Pr[\;\AAA' \;\text{succeeds}\;\wedge x_i \not\in G\;] \\
&\le \Pr_x[\;\text{every}\;x_i \in G_i\;]
 + \sum_{i=1}^m \Pr[\;\AAA' \;\text{succeeds}\;|\;x_i \not\in G\;] \\
&\le (1-\frac{\delta}{2})^{2n/\delta} + \sum_{i=1}^m \frac{\alpha}{2m} \\
&\le 2^{-n} + \frac{\delta}{2} < \delta
\end{align*}

Construct $\AAA$ against $f$:
$\AAA(\texttt{1}^n, y=f(x_i))\eqdef$ Repeat: Choose all of the $x_j$ for $j \neq i$,
invoke $\AAA'(f'(x')))$, win if $\AAA'$ succeeds.

\[ \Pr_{x_i}[\;\AAA \; \text{inverts}\;] \approx \Pr_{x_i}[x_i\;\text{is good}] \ge 1-\delta/2 \]

\paragraph{Family of one-way functions}
$F = \{ f_s : D_s \rightarrow R_s \}$
\begin{itemize}
\item Easy to sample a function: $\exists$ $ppt$ $S(\texttt{1}^n)$ outputs some $f_s \in F$
\item Easy to sample from $D_s$
\item Easy to evaluate $f_s(x)$
\item Hard to invert: $\forall$ $\nuppt$ $\AAA$,
\[ \Adv_F(\AAA) = \Pr_{\substack{s\gets S(\texttt{1}^n),\\x\gets D_s}} [\;
  \AAA(\texttt{1}^n, s, f_s(x)) \in f_s^{-1}(f_s(x))
\;] = \negl(n) \]
\end{itemize}

\paragraph{Family of subset-sum functions}
\[F_{\text{ss}} = \left\{ f_{\vec{a}}(b_1,\ldots,b_n) = \sum_i a_i\cdot b_i \bmod N \right\} \]

\paragraph{Rabin's function}
$f_N : \ZZ_N^* \rightarrow \ZZ_N^*$,
$N=pq$ for distinct primes $p$, $q$,
$f_N(x)=x^2 \bmod N$

\paragraph{Rabin OWF family $\iff$ factoring is hard}

Chinese remainder theorem:
$h: \ZZ_N \to \ZZ_p \times \ZZ_q$\\
$p,q$ distinct primes, $N=pq$\\
$h(x) = (x \bmod p, x \bmod q)$ \\
Ring homomorphism: $h(x\cdot y) = h(x) \cdot h(y)$, $h(x+y) = h(x) + h(y)$\\
... is an isomorphism (a bijection)

$\abs{\QR_p^*}=(p-1)/2$ for any odd prime $p$:
because $a^2 = (-a)^2 = (p-a)^2 \mod p$.
$\ZZ_p$ is a field so there are $\le 2$ square roots for any element.
Therefore $x \rightarrow x^2 \mod p$ is a 2-to-1 function $\ZZ_p^* \rightarrow \QR_p^*$

$\ZZ_N^* \cong \ZZ_p^* \times \ZZ_q^*$, so $\QR_N^* \cong \QR_p^* \times \QR_q^*$
$\implies$ $\abs{QR_N^*} = \frac{p-1}{2}\frac{q-1}{2} = \frac{\phi(N)}{4}$ \\
For $(a_p, a_q) \in \QR_p^*\times\QR_q^*$, $(\pm\sqrt{a_p},\pm\sqrt{a_q})^2 = (a_p,a_q)$,
so Rabin is 4-to-1.

Lemma: Given any $x_1, x_2 \in \ZZ_N^*$ such that $x_1^2 = x_2^2$ but $x_1 \neq \pm x_2 \mod N$,
the factorization of $N$ can be computed efficiently.
Proof: $x_1^2 = x_2^2 \mod p$ and $\mod q$.
So $x_1=\pm x_2 \mod p$ and $x_1=\pm x_2 \mod q$, but the $\pm$ for these
two statements cannot both be $+$ or both $-$.
Otherwise, by Chinese remainder theorem, $x_1 = x_2$.
w.l.o.g. $x_1 = +x_2 \mod p$ and $x_1 = -x_2 \mod q$.
So $p \divides (x_1-x^2)$ and $q\nmid(x_1-x_2)$,
and $\gcd(N, x_1-x_2) = p$.

If factoring is easy: $\AAA(N=pq, y=x^2\mod N)\eqdef$
Factor $N$ into $p,q$.
Compute $\sqrt{y\bmod p}\in\ZZ_p^*$, $\sqrt{y\bmod q}\in\ZZ_q^*$ (proof omitted).
Reconstruct $\sqrt{y}\in\ZZ_N^*$ using Chinese remainder theorem bijection (proof omitted).

If Rabin inversion is easy:
Assume $\exists$ $\BBB$ for which
$\Pr_{N\gets pq,X\gets\ZZ_N^*}[\BBB(N, y=x^2\bmod N)\in\sqrt{y}] \ge \alpha(n)$ non-$\negl(n)$.
Build $\AAA(N=pq)$: Choose $x\gets\ZZ_N^*$, let $y=x^2\mod N$.
$\BBB(N, y) = x' \in \ZZ_N^*$ such that $(x')^2 = y \mod N$ (w.p. $\ge\alpha(n)$).
If $x \neq \pm x' \mod N$ (w.p. $1/2$), we can use the lemma.
Overall, succeeds w.p. $\ge \alpha(n)/2$.

\section{Psuedo-random generators (PRGs)}

If $\exists$ a PRG $G$ with output length $n+1$, $\exists$ a PRG $G_\ell$ with
output length $\ell(n) = \poly(n)$.

\section{Pseudo-random functions}

\paragraph{Uniformly random function}
$U$ is a random variable chosen uniformly from the set of all functions
$\{0,1\}^m \rightarrow \{0,1\}^n$.

\paragraph{Pseudo-random function}
A PRF belongs to a family of functions
$F : \{0,1\}^\ell \times \{0,1\}^m \rightarrow \{0,1\}^n$.
Write $F_k(\cdot)$ to denote $F(k, \cdot)$.

\paragraph{Distinguishing advantage}
Consider an adversary $\mathcal{A}$ who knows $F$, having oracle access
to $F_k$ where $k$ was chosen uniformly at random, trying to
distinguish the oracle's responses from a random function.
The distinguishing advantange of $\mathcal{A}$ against $F$ is
\[ \textrm{Adv}_F^\textrm{prf}(\mathcal{A}) \eqdef
\Pr_{k\in\{0,1\}^\ell} \big[\, \mathcal{A}^{F_k}\;\textrm{accepts} \,\big]
- \Pr_U \big[\, \mathcal{A}^U \textrm{ accepts} \,\big] \;\textrm{.} \]

In time $O(t)$, we can brute-force $t$ keys to get
advantage $t / 2^\ell$.

\paragraph{$(t,q)$-bounded adversary}
\begin{tabular}{r|l}
$t$ & Running time \\
$q$ & Number of queries
\end{tabular}

\paragraph{$(t,q,\eps)$-secure PRF}
$F$ is $(t,q,\eps)$-secure iff
$\forall$ $(t,q)$-bounded $\mathcal{A}$,
\[ \textrm{Adv}_F^\textrm{prf}(\mathcal{A}) \le \eps \;\textrm{.} \]

\paragraph{Examples of reasonable constants}
\begin{tabular}{r|l}
$t$ & $2^{128}$ \\
$q$ & $2^{64}$ or $2^{32}$ \\
$\eps$ & $2^{-128}$
\end{tabular}

\paragraph{Existence}
The existence of secure PRFs has not been proven,
but there are some functions that have never been
broken and are widely assumed to be PRFs.

\section{Goldreich-Levin}

Given $\overline{F}(x)$ OWF, define $F(x,r)\eqdef(\overline{F}(x), r)$ for $\abs{x}=\abs{r}$.
Then \[ h(x,r)\eqdef \ang{x,r} \bmod 2 = \sum_{j=1}^n x_j \cdot r_j \bmod 2 \]
is hard-core for $F$.

Suppose $\exists$ $\AAA$ and non-negl $\delta$ s.t. \[
\Pr_{x,r\gets\bit^n}[\;\AAA(\overline{F}(x), r) = \ang{x,r}\;] \ge \frac{1}{2} + 2\delta
\]
Then we'll construct $\BBB$ s.t. \[
\Pr_{x\gets\bit^n}[\;\BBB(\overline{F}(x)) \in \overline{F}^{-1}(\overline{F}(x))\;] \ge \delta
\]

Let $\OOO(\cdot) \eqdef \AAA(y, \cdot)$.

Define SC($z$, $\{r\}_m\in\bit^n$, $\{b\}_m\in\bit$) for $m = O(\log n)$.
Assume that $b_j=\ang{x,r_j}$ (we will end up trying all the $\{b\}$).

For each subset $S \subseteq [m]$: Let $b_S = \sum_{j\in S} b_j$ and $r_s = \sum_{j\in S} r_j$.
So $\ang{x,r_s} = \ang{x,\sum_{j\in S} r_j} = \sum_{j\in S}\ang{x,r_j} = \sum_{j\in S} b_j = b_S$.
Then $c_S \gets \OOO(z+r_s) - b_S$. Return the majority value of $c_S$.

$\BBB$:
\begin{quote}
For each $j\in[m]$ choose $r_j\gets\bit^n$ randomly.

For each $\{b\}\in\bit^m$:
\begin{quote}
For each $k\in[n]$ let $x'_k \gets$ SC$(e_k, \{r\}_m, \{b\}_m)$

Let $x' = x'_1 \ldots x'_n$. Output $x'$ if it is an inversion.
\end{quote}
\end{quote}

Over random choice of the $r_j$, the $r_S$ are uniformly random
and pairwise independent for all $S \neq \phi$.

$X_S \eqdef \text{ind}(\OOO(z+r_s)\;\text{is correct}(=\ang{x,z+r_s}))$

We're interested in whether $X=\sum_{S\subseteq[m]} X_S > \frac{m}{2}$

\paragraph{Local list-decoding the Hadamard code}
With a weak SC and $<\frac{1}{4}$ fraction of errors, can recover the unique $x$.
GL says with strong SC at $\frac{1}{2}-\delta$ fraction of errors, we can
recover all possible $x$.

\section{Reduction}

\paragraph{Karp (many-to-one) reduction}
Reduction from $A$ to $B$ transforms
an instance of $A$ to an instance of $B$.

\paragraph{Cook (Turing) reduction}
Reduction from $A$ to $B$ solves $A$
using a subroutine that solves $B$.

\paragraph{Key recovery security}
$F$ is $(t,q,\eps)$-kr-secure
iff $\forall$ $(t,q)$-bounded $\mathcal{A}$,
\[ \textrm{Adv}_F^\textrm{kr} \eqdef
\Pr_{k \in \{0,1\}^\ell} \big[\,
  \mathcal{A}^{F_k(\cdot)}\;\textrm{outputs}\;k
\,\big] \le \eps \;\textrm{.} \]

\begin{thm}
If $F$ is a $(t,q,\eps)$-secure PRF
for $q < 2^m$, then
$F$ is $(t',q',\eps')$-kr-secure for
$t' \approx t$, $q' = q-1$, $\eps' = \epsilon + 2^{-n}$.
\begin{proof}
Cook reduction.
For any kr-adversary $\mathcal{A'}$ running in
time $t'$ and making $q'<2^m$ queries, let $\mathcal{A}$ be
the PRF adversary:
\begin{quote}
$k' \; \leftarrow \; \mathcal{A}'(\mathcal{O})$ \\
$x \; \leftarrow$ a value that $\mathcal{A}'$ did not query with \\
$y \; \leftarrow \; \mathcal{O}(x)$ \\
Accept iff $y = F_{k'}(x)$
\end{quote}
$\mathcal{A}$ runs in time $t \approx t'$ and makes $q = q'+1$ queries.
\[ \textrm{Adv}_F^\textrm{prf}(\mathcal{A}) \ge
\textrm{Adv}_F^\textrm{kr}(\mathcal{A}') - 2^{-n}
\;\textrm{.} \qedhere \]
\end{proof}
\end{thm}

\paragraph{Example PRF construction}
For PRF $F : \bit^\ell \times \bit^n \rightarrow \bit^n$,
$F'_k(x) \eqdef F_k(F_k(x)) \Vert F_k(\overline{F_k(x)})$

\begin{thm}
$F'$ is $\displaystyle (t, \frac{q}{3}, \eps + \frac{q^2}{2^n})$-secure.
\begin{proof}
Let $\mathcal{A}'$ be an attacker on $F'$.
Define $\mathcal{A}$ as:
\begin{quote}
$\mathcal{O}' \eqdef
\mathcal{O}(\mathcal{O}(x)) \Vert \mathcal{O}(\overline{\mathcal{O}(x)})$
(done with $3$ queries to $\mathcal{O}$)

Accept iff $\mathcal{A}'^{\,\mathcal{O}'}$ accepts
\end{quote}
$\mathcal{O}'(x)$ simulates $F'$ perfectly, so
$\displaystyle
\Pr_k \big[\, \mathcal{A}^{F_k} \,\big] =
\Pr_k \big[\, \mathcal{A'}^{\,F'_k} \,\big]$.

$\mathcal{O}'$ does not simulate $U$ perfectly, but it is close.
We have independence as long as all of the
$\mathcal{O}(x)$, $\overline{\mathcal{O}(x)}$ are distinct.
Using union bound, this probability $\le \frac{q^2}{2^n}$
\end{proof}
\end{thm}

\section{Pseudo-random permutations}

In a permutation family
$F : \{0,1\}^\ell \times \{0,1\}^n \rightarrow \{0,1\}^n$,
every $F_k$ is bijective.

A secure PRP is computationally indistinguishable from
a uniformly random permutation.

\paragraph{Some well-known PRPs}
\begin{tabular}{l|lll}
$\mathsf{DES}$ & $\ell = 56$ & $n = 64$ \\
$\mathsf{AES_{128}}$ & $\ell = 128$ & $n = 128$ \\
$\mathsf{AES_{192}}$ & $\ell = 192$ & $n = 128$
\end{tabular}

\paragraph{Strong PRP / block cipher}
Attackers with oracle access to both $F$ and $F^{-1}$
have small advantage. \[
\text{Adv}_F^\text{sprp} \eqdef
  \Pr_k \big[\, \mathcal{A}^{F_k,F_k^{-1}} \; \text{accepts} \,\big]
- \Pr_P \big[\, \mathcal{A}^{P,P^{-1}} \; \text{accepts} \,\big]
\le \eps
\]

\paragraph{PRF/PRP switching lemma}
If $G$ is a $(t, q, \eps)$-secure PRP (not necessarily strong),
then $F$ is a $(t, q, \eps + \frac{q^2}{2^{n+1}})$-secure PRF.

\section{Secure symmetric encryption}

Perfect secrecy is impossible where $m > \ell$, but
computational security is possible with pseudorandom objects.

\paragraph{Electronic code block (ECB)}
Suppose $F$ is a secure PRP
$\bit^\ell\times\bit^n\rightarrow\bit^n$
with $F$ and $F^{-1}$ efficiently computable.
~\\[5pt]
\begin{tabular}{r|l}
\Gen & $k \leftarrow \bit^\ell$ \\[5pt]
\Enc & $M'$ $\leftarrow$ Pad message $M$
       with \texttt{1} and some \texttt{0}s
       to a multiple of $n$. \\
     & Break $M'$ into $n$-bit blocks $m_0, m_1, \ldots$ \\
     & Apply $F_k$ to each of the $\{m\}$ \\[5pt]
\Dec & Apply $F'_k$ to each of the $\{m\}$
\end{tabular}\\\\
Repeated blocks give repeated ciphertext.
Never use ECB.

\paragraph{Security model}
Adversary, seeing all cipthertexts and
having oracle access to $\Enc_k$,
learns nothing about plaintexts
(except message length, which is unavoidable).

$SE = (\Gen, \Enc, \Dec)$ is
$(t, \sigma, \eps)$-IND-CPA secure
(``indistinguishable under chosen-plaintext attack'')
iff $\forall$ $(t, \sigma)$-bounded $\mathcal{A}$, \[
\text{Adv}_{SE}^\text{indcpa}(\mathcal{A}) \eqdef
\Pr_{k \leftarrow \Gen} \big[\, \mathcal{A}^{L_k} \; \text{accepts} \,\big]
- \Pr_{k \leftarrow \Gen} \big[\, \mathcal{A}^{R_k} \; \text{accepts} \,\big]
\;\text{,}
\]
\[ L_k(m, m') \eqdef \Enc_k(m)
\;\text{if}\; |m| = |m'|\;\text{else}\; \bot \;\text{,} \]
\[ R_k(m, m') \eqdef \Enc_k(m')
\;\text{if}\; |m| = |m'|\;\text{else}\; \bot \;\text{,} \]
$t$ is the running time, and
$\sigma$ total length of all message queries.

Equivalent definition:
$\Enc_k$ is computationally indistinguishable from
a zero-encrypting oracle $Z_k \eqdef \Enc_k(\texttt{0}^m)$.

\paragraph{Query repetition}
\Enc\ in an IND-CPA-secure scheme should not always return the same ciphertext
for multiple encryptions of the same message.
This attack has advantage $1$ against any deterministic and stateless scheme:
\\~\\
\begin{tabular}{|l}
$c \leftarrow \mathcal{O}(\ang{0}, \ang{0})$\\
$c' \leftarrow \mathcal{O}(\ang{0}, \ang{1})$\\
Accept iff $c = c'$
\end{tabular}

\section{Block cipher modes}

\paragraph{Stateful counter mode (CTRS)}
Let $F$ be a PRF with $m = n$.
~\\[5pt]
\begin{tabular}{r|l}
\Gen & $k \leftarrow \bit^\ell$, $counter \leftarrow 0$ \\[5pt]
\Enc & echo $counter$ \\
     & for each message block $m$: \\
     & ~~~~echo $F_k(counter) \oplus m_i$ \\
     & ~~~~increment $counter$
\end{tabular}\\

CTRS is not used much, because preserving $counter$ is difficult.

\begin{thm}
If $F$ is a $(t,q,\eps)$-secure PRF, then
CTRS$(F)$ is $(t'\approx t, qn, 2\eps)$-IND-CPA secure.
\begin{proof}

We will show using a hybrid argument that
$\forall$ $(t',\sigma)$-bounded $\mathcal{A}'$ against CTRS$(F)$
where $\sigma \le n\,2^m$, there is a
$(t\approx t', q=\sigma/n)$-bounded attacker $\mathcal{A}$
attacking $F$ such that
$\text{Adv}_\text{CTRS$(F)$}^\text{indcpa}(\mathcal{A}')
\le 2\,\text{Adv}_F^\text{prf}(\mathcal{A})$.

Given $\mathcal{O}$ that is either $F$ or $U$:\\

$\mathcal{A} \eqdef \mathcal{A}_L \eqdef$
\begin{tabular}{|l}
$counter$ $\leftarrow$ $0$\\
$\mathcal{O}'(m,m') \eqdef$
\begin{tabular}{|l}
If $|m|=|m'|$, return $\bot$\\
Split $m$ into blocks $m_0, m_1, \ldots, m_{t-1}$\\
$y_i$ $\leftarrow$ $\mathcal{O}(counter + i)$ $\forall$ $i\in[0,t)$\\
Return $counter\Vert \text{join}_i(m_i \oplus y_i)$\\
$counter$ $\leftarrow$ $counter + t$
\end{tabular}\\
Accept iff $\mathcal{A}'^{\,\mathcal{O}'}$ accepts
\end{tabular}

Also define $\mathcal{A}_R$ similarly using $m'$ instead of $m$.

$\mathcal{A}_L^{F_k}$ perfectly simulates $L_k$ to $\mathcal{A}'$.

$\mathcal{A}_L^U$ does not simulate $R_k$, but it does simulate an oracle $\$$:
\\~\\
$\$(m,m') \eqdef$
\begin{tabular}{|l}
If $|m|=|m'|$, return $\bot$\\
Return $counter\Vert[\text{random bits}]$\\
$counter$ $\leftarrow$ $counter + \text{number of blocks}$
\end{tabular}

\begin{align*}
P_\ell &= \Pr_k[\mathcal{A}_L^{F_k}]= \Pr_k[\mathcal{A}'^{\,L_k}] \\
P_r &= \Pr_k[\mathcal{A}_R^{F_k}]= \Pr_k[\mathcal{A}'^{\,R_k}] \\
P_\$ &= \Pr_k[\mathcal{A}_L^U]=\Pr_k[\mathcal{A}_R^U]= \Pr_k[\mathcal{A}'^{\,\$}]
\end{align*}
\begin{align*}
\text{Adv}_\text{CTRS$(F)$}^\text{indcpa}(\mathcal{A}') &= |P_\ell - P_r| \\
&\le |(P_\ell - P_\$) + (P_r - P_\$)| & \text{(triangle inequality)} \\
&\le \eps+\eps = 2\eps
\end{align*}

\end{proof}
\end{thm}

\paragraph{Counter modes}~\\

\begin{tabular}{l|l|l}
CTRS & One global counter &
$\text{adv}^\text{indcpa} \le 2\,\text{adv}^\text{prf}$ \\[5pt]
CTR\$ & Random IV for each message &
$\text{adv}^\text{indcpa} \le 2\,\text{adv}^\text{prf}+q^2/2^n$ \\[5pt]
CTR\$\$ & Random IV for each block
\end{tabular}

\paragraph{Cipher block chaining (CBC)}

$C_0 = \text{IV}$,
$C_i = F_k(C_{i-1}\oplus m_i)$

\Dec\ requires being able to calculate $F^{-1}$.

If $F$ is a $(t,q,\eps)$-secure PRF, then
CBC[$F$] is $(\approx t, \sigma=qn, 2\eps+q^2/2^n)$-ind-cpa-secure.
The proof requires showing that for $U$, all inputs to $U$
are distinct (minus a birthday term).

\section{Message authentication code (MAC)}

Alice sends message $m$ and $t \leftarrow \Tag_k(m)$.
Eve intercepts $(m,t)$ and delivers $(m',t')$ to Bob.
Bob runs $\Ver_k(m',t')$.

$\Ver_k$ returns $\begin{cases}
m & \text{if $t'$ is a valid tag ($\Ver_k$ ``accepts'')} \\
\bot & \text{otherwise ($\Ver_k$ ``rejects'')}
\end{cases}$

Eve has access to a $\Tag_k$ oracle and can
make many attempts on $\Ver$.
Eve ``wins'' if \Ver\ accepts on an $m'$
not previously queried to $\Tag_k$.

\paragraph{Conerns ignored by this model}
\begin{itemize}
\item Dropped messages
\item Replay attacks (``freshness'' of messages)
\item Message sequence
\end{itemize}

\paragraph{Unforgeability under chosen message attack}

\[ \text{Adv}_\text{MAC}^\text{ufmca}(\mathcal{A}) \eqdef
\Pr_{k\leftarrow\Gen} \left[\;\mathcal{A}^{\Tag_k,\Ver_k}\;\text{``wins''}\;\right] \]

MAC is $(t, q_t, q_v, \eps)$-uf-cma-secure iff
advantage of an attacker
bounded by time $t$,
number of $\Tag$ queries $q_t$, and
number of $\Ver$ queries $q_v$
is less than $\eps$.

\paragraph{Examples of reasonable constants}
\begin{tabular}{r|l}
$t$ & $2^{80}$ or $2^{128}$ \\
$q_v,q_t$ & $2^{40}$ or $2^{56}$ \\
$\eps$ & $2^{-40}$ or $2^{-56}$
\end{tabular}

\paragraph{Brute-force MAC attacks}

\begin{itemize}
\item Key search:
Get a few oracle tags, and guess $k$.
$\text{Adv} = t/2^\ell$.
\item Tag search:
$\text{Adv} = t/2^s$ where $s$ is the tag length.
\end{itemize}

\paragraph{PRF-based MAC}
$\Tag_k \eqdef F_k$

$\forall$ $(t,q_t,q_v)$-bounded $\mathcal{B}$,
$\exists$ $(\approx t, q_t+q_v)$-bounded $\mathcal{A}$
such that \[ \text{Adv}_\text{PRFMAC[$F$]}^\text{ufcma}(\mathcal{B})
\le \text{Adv}_F^\text{prf}(\mathcal{A}) + q_v/2^n \;\text{.} \]

\paragraph{CBC-MAC}

For a fixed $t$, and $F:\bit^{nt}\rightarrow\bit^n$, CBC-MAC[$F$] is secure,
losing $(qt)^2/2^n$ advantage from that of $F$.

\paragraph{Cipher-based MAC (CMAC)}
Adds an extra step to the end of CBC-MAC
to make it secure for arbitrary-length messages.

Precompute $k_1$, $k_2$ $\in$ $\bit^n$ using $F_k(\texttt{0}^m)$.\\

$m'_t \leftarrow \begin{cases}
m'_t\oplus k_1 &: |m'_t|=n \\
m' \Vert \texttt{000}\ldots \oplus k_2 &: |m'_t|<n
\end{cases}$\\

Run $m_1 \Vert \ldots \Vert m_t$ through CBC-MAC.

\section{Combining authenticity and privacy}

\paragraph{Integrity of ciphertexts (INT-CTXT)}
$\Dec_k(c)$: returns decryption of $c$, or $\bot$ if $c$ is invalid.

$SE=(\Gen,\Enc,\Dec)$ is INT-CTXT secure iff $\forall$ bounded $\mathcal{A}$,
\[ \text{Adv}_{SE}^\text{int-ctxt}(\mathcal{A}) \eqdef
\Pr_{k\leftarrow\Gen} \left[\;\mathcal{A}^{\Enc_k,\Dec_k}\;\text{wins}\;\right]
< \eps \;\text{.} \]

UF-CMA-security does not necessarily give INT-CTXT security.
For example: If the output of $\Tag$ has a spurious bit that is
ignored by $\Ver$. So we require a stronger condition:

\paragraph{Strong unforgeability (SUF-CMA)}
Winning is redefined as: $\Ver_k$ accepts $(m',t')$
that was not previously a query/answer pair to $\Tag_k$.

\paragraph{Bad idea: Encrypt-and-tag}
$\AEnc \eqdef \EEnc_{k_e}(m) \Vert \Tag_{k_m}(m)$.
The tag could reveal information about $m$.

\paragraph{Bad idea: Tag-then-encrypt}
$\AEnc \eqdef \EEnc_{k_e}( m \Vert \Tag_{k_m}(m) )$.
The ciphertext might be forgeable (for example,
if $\EEnc$ appends a spurious bit).

\paragraph{Good idea: Encrypt-then-tag}
$\AEnc \eqdef \EEnc_{k_e}(m) \Vert \Tag_{k_m}(\EEnc_{k_e}(m))$.

\paragraph{Indistinguishability under chosen ciphertext attack (IND-CCA)}
IND-CPA $\wedge$ INT-CTXT $\Rightarrow$ IND-CCA.

\section{Summary of symmetric crypto games}

\begin{tabular}{ll|l|l}
&& Oracles & Goal \\ \hline
\multirow{2}{*}{\begin{sideways}$\xleftarrow{\text{Stronger}}$\end{sideways}}
& IND-CPA & L/R & \multirow{2}{*}{Left or right?} \\
& IND-CCA & L/R, $\hat{\mathsf{Dec}}$ \\ & & \\ \hline
\multirow{2}{*}{\begin{sideways}$\xleftarrow{\text{Stronger}}$\end{sideways}}
& INT-PTXT & \multirow{2}{*}{$\mathsf{Enc}$, $D$} & $D$ returns new plaintext \\
& INT-CTXT & & $D$ returns non-$\bot$ on new ciphertext \\ & & \\ \hline
\multirow{2}{*}{\begin{sideways}$\xleftarrow{\text{Stronger}}$\end{sideways}}
& UF-CMA & \multirow{2}{*}{$\mathsf{Tag}$, $\mathsf{Ver}$}
& $\mathsf{Ver}(m,t) :$ $m$ has never been tagged \\
& SUF-CMA &
& $\mathsf{Ver}(m,t) :$ $(m,t)$ has never been a $\mathsf{Tag}$ pair \\ \\
\end{tabular}

\section{Hashing}

\paragraph{Hash function} $h:D\rightarrow \bit^n$, $D > 2^n$

\paragraph{Collision} $x,x' \in D : h(x)=h(x') \wedge x \neq x'$

\paragraph{Hash family} $H : \bit^\ell \times D \rightarrow \bit^n$

\paragraph{Collision resistance (CR)}
$H$ is $(t,\eps)$-collision resistant if
$\text{Adv}_H^\text{cr}(\mathcal{A}) \le \eps$
$\forall$ $t$-bounded $\mathcal{A}$.
\[ \text{Adv}_H^\text{cr}(\mathcal{A}) = \Pr_{k\leftarrow\bit^\ell}
\left[\;\mathcal{A}(k)\;\textrm{outputs a collision in}\;H_k\;\right] \]

\paragraph{Real-world hash functions}
\begin{tabular}{lrl}
MD4 & $n=128$ & Broken \\
MD5 & $n=128$ & Broken \\
SHA-1 & $n=160$ & Maybe broken \\
SHA-256 & $n=256$ & Good \\
SHA-3 & & Good
\end{tabular}

Hash output lengths need to be longer than encryption key lengths
because brute-force attacks can test $\approx q^2$ pairs in $q$ hashes
(``birthday attack'').

\paragraph{Second preimage resistance / target collision resistance (TCR)}
Given $x$, attacker must find $x'$ such that $x,x'$ is a collision.
\[ \text{Adv}_H^\text{tcr}(\mathcal{A}) =
\Pr_{\substack{k\leftarrow\bit^\ell,\\x\leftarrow D}}
\left[\;\mathcal{A}(k, x)\;\text{outputs a collision}\;\right] \]
where $D$ is some distribution over the message space.

Brute force attack does not have a birthday advantage in this game.

CR $\Rightarrow$ TCR.

\paragraph{One-wayness (OW)}

\[ \text{Adv}_H^\text{ow}(\mathcal{A}) =
\Pr_{\substack{k\leftarrow\bit^\ell,\\x\leftarrow D}}
\left[\;\mathcal{A}(k, H_k(x))\;\text{outputs}\;x':H_k(x')=H_k(x)\;\right] \]

TCR $\Rightarrow$ OW for ``high-entropy'' $D$
(so that $H_k(x)$ reveals very little about $x$).

\paragraph{Merkle-Damg{\aa}rd (MD) transform}
$\text{MD}[h]_k(M\in\bit^*)$

Uses a compression function $h_k:\bit^\ell\times\bit^{b+n}\rightarrow\bit^n$

Break $M$ into $M_1,\ldots,M_t$ s.t. $\|M_i\|=b$

\begin{align*}
y_1 &\leftarrow h_k(\;M_1\,\Vert\,\ang{0}\;) \\
y_2 &\leftarrow h_k(\;M_2\,\Vert\,y_1\;) \\
&\vdots\\
y_i &\leftarrow h_k(\;M_i\,\Vert\,y_{i-1}\;) \\
&\vdots\\
y_t &\leftarrow h_k(\;M_t\,\Vert\,y_{t-1}\;) \\
y &\leftarrow h_k(\;\ang{t}\,\Vert\,y_t\;)
\end{align*}

\paragraph{If $h$ is CR, then MD[h] is CR}
Let $\mathcal{B}$ attack MD$[h]$ in CR game.
We can build $\mathcal{A}$ attacking $h$ in CR game.
$\mathcal{A}(k\in\bit^\ell)$ will use a collision
$x,x'$ in MD$[h_k]$ to find a collision in $h_k$.

If $x,x'$ have different numbers of blocks:
Then $\ang{t}\,\Vert\,M_t, \ang{t'}\,\Vert\,M'_t$ is a collision in $h_k$.

Otherwise: Walk backward through the MD process to find a step
where $M_i\,\Vert\,y_{i-1} \neq M'_i\,\Vert\,y'_{i-1}$.

\paragraph{HMAC} Secure MAC based on hash function.

$\text{HMAC}_k(m) = H(
  (k \oplus \text{opad})
  \,\Vert\,
  H(
    (k \oplus \text{ipad})
    \,\Vert\,
    m
  )
)$
where $k$ is the MAC key padded to the length of the compression function,
ipad and opad are fixed strings of the same length.

Only TCR security of $H$ is required for the security of HMAC.

\section{Groups}

\paragraph{Group axioms}

\begin{itemize}
\item $G$ is closed under $\cdot$
\item $\I\cdot a=a\cdot\I=a$
\item Every element of $G$ has a unique multiplicative inverse
\item $(a\cdot b)\cdot c=a\cdot(b\cdot c)$
\end{itemize}

The group operator $\cdot$ is not necessarily commutative.

\paragraph{Examples of groups}

\begin{itemize}
\item Integers under addition
\item Invertible matrices with real entries under multiplication
\item $\ZZ_N=\{0,1,\ldots,N-1\}$ under modular addition
\item $\ZZ_N^*=\{a\in\ZZ_N:\gcd(a,N)=1\}$ under multiplication $(\bmod\;N)$
\end{itemize}

\paragraph{$\ZZ_N^*$ is closed under $\cdot$} Proof sketch:\\
$\gcd(a\cdot b,N)=1 \implies \gcd(ab\bmod N,N)=1$.

\paragraph{$\ZZ_N^*$ has unique inverses} Proof sketch:\\
Use the fact $\gcd(a,N)=1 \iff \exists\;x,y\in\ZZ:ax+Ny=1$.\\
$ax=1-Ny = 1 \mod N \implies x = a^{-1} \mod N$.

\paragraph{Generator} $g\in G: \{g^0,g^1,g^2,\ldots\} = G$

If $G$ is finite, then $a^\abs{G}=1\;\forall\;a\in G$.

$g$ is a generator iff $g^i\neq\I\;\forall\;0<i<\abs{G}:i\divides G$

For a prime $p$, $\ZZ_p^*$ is finite and cyclic (has a generator).
Its order is $p-1$.

\paragraph{Order} of $a\in G$ is $(\min i>0 : a^i=1)$.\\
order$(a)$ divides $\abs{G}$.

\paragraph{How many generators?}
If $g\in G$ is a generator and $i$ is coprime with $\abs{G}$,
then $g^i$ is a generator.
Therefore we have $\abs{\ZZ_{p-1}^*}$ generators.

\section{Diffie-Hellman protocol (1976)}

For establishing a secret key without meeting.

$A$ chooses $a \in G$ and transmits $g^a$.
$B$ chooses $b \in G$ and transmits $g^b$.
The secret is $g^{ab}$.

Requires a multiplicative group $(G,\cdot)$
and a generator $g\in G$,
and the discrete log problem must be intractable.

\paragraph{Calculations required for implementation}
Multiply huge numbers $(\approx 2^{4096})$,
exponentiate huge numbers,
sample huge primes, find a generator of $\ZZ_p^*$.

\paragraph{Multiplication}
Simple grade-school multiplication algorithm
followed by Euclid's division algorithm.
Both are quadratic in bit length.

\paragraph{Exponentiation}
Repeated squaring. $O(\lg^3 a)$ multiplications.

\paragraph{Primality testing}
Miller-Rabin test runs in $O(\lg^3p)$, randomized with false positives.
Alternately, AKS is a deterministic test running in $O(\lg^6p)$.
Approximately $1/k$ of $k$-bit numbers are prime.

\paragraph{Generator testing}
$g$ is a generator if $g^{(p-1)/i}=1$ $\forall$
prime divisors $i$ of $p-1$.
Since factoring is hard, we must generate $p$ such that we know
the factors of $p-1$ \ldots

\paragraph{Sophie Germain (safe) primes}
Pick a prime $q$ and let $p=2q+1$. Repeat until $p$ is prime.
It appears that $1/k^2$ of $k$-bit numbers are safe primes,
although this is unproven.

\paragraph{Uniformity of shared secret}
We want $ab\bmod(p-1)$, and therefore also $g^{ab}$,
to be uniformly distributed.
If $G=\ZZ_p^*$, this is not the case;
$ab$ is even with probability $3/4$, and the $\bmod(p-1)$
operation does not affect this parity (because $p-1$ is even).
But if $G$ has prime order $q$, then $ab\bmod q$ is very nearly uniform
($ab=0$ with probability $2/q$, anything else with probabilility $1/(q-1)$).

\paragraph{Quadratic residue}
The subgroup of quadratic residues in $\ZZ_p^*$ is
$\QR_p^* = \{(g^2)^0, (g^2)^1, (g^2)^2, \ldots, (g^2)^{(p-3)/2}\}$.
For $p=2q+1$, $\abs{\QR_p^*}=q$, so if $p$ is a safe prime,
$\abs{\QR_p^*}$ has prime order.
$g^2$ is a generator for $\QR_p^*$.

\paragraph{Jacobi symbol} For $y\in\ZZ_p^*$, Jacobi symbol is
$1$ if $y\in\QR_p^*$, and $-1$ otherwise.
$y\in\QR_p^*\iff y^{(p-1)/2}=1$.
Proof:
$y^{(p-1)/2}=g^{i(p-1)/2}$ for some $i=2i'+b$, $b\in\{0,1\}$.
Use the fact that $g^{(p-1)/2}=-1$.
$y=g^{b(p-1)/2} = \{ 1 \text{ if } b=0\text{, } -1\text{ if } b=1 \}$.

\section{Asymmetric encryption}

\paragraph{Scheme $\AE=(\Gen,\Enc,\Dec)$}
\begin{itemize}
\item $(\pk,\sk)\gets\Gen$
\item $c\gets\Enc_\pk(m)=\Enc(\pk,m)$ where
      $m\in M_\pk$, and $M_\pk$ is some group
\item $m\text{ or }\bot\gets\Dec_\sk(c)$
\item \Gen\ and \Enc\ are randomized. \Dec\ is deterministic.
\end{itemize}

\paragraph{IND-CPA-security}
\[\Adv_\AE^\indcpa(\mathcal{A})\eqdef
\Pr_{(\pk,\sk)\gets\Gen}\left[\mathcal{A}^{L_\pk}(\pk)\right] -
\Pr_{(\pk,\sk)\gets\Gen}\left[\mathcal{A}^{R_\pk}(\pk)\right]
\] where $L_\pk(m_L,m_R)=\Enc_\pk(m_L)$
and $R_\pk(m_L,m_R)=\Enc_\pk(m_R)$.

\paragraph{IND-CCA-security}
Like IND-CPA, but the attacker gets another oracle:
\[ D_\sk(c)=\begin{cases}
\Dec_\sk(c)&\text{if $c$ was not returned by L/R oracle}\\
\bot&\text{otherwise}
\end{cases}\text{ .} \]

\paragraph{One encryption query}
$\forall$ ind-cpa attacker $\mathcal{B}$ against AE,
$\exists$ an ind-cpa attacker $\mathcal{A}$ against AE
making $\le q\; \Enc$ queries
such that $\Adv_\AE^\indcpa(\mathcal{B}) \le q\,\Adv_\AE^\indcpa(\mathcal{A})$.
(Also, same for ind-cca with $1$ encryption query and unlimited $D_\sk$ queries.)
The analogous claim is not true in a symmetric key setting.

\paragraph{ElGamal}
Asymmetric encryption scheme similar to Diffie-Hellman.
\begin{itemize}
\item $\Gen$: Choose group $G$ of order $n$, generator $g$.
      Choose $x\gets\ZZ_n$. Output $(\sk=x,\pk=(G,G,X=g^x))$.
\item $\Enc_\pk(m\in G)$: Choose $y\gets\ZZ_n$.
      Output $c=(Y=g^y, \overbrace{m\cdot X^y}^{g^{xy}})$.
\item $\Dec_\sk(\overbrace{Y}^{g^y},\overbrace{Z}^{m\cdot g^{xy}})$:
      Output $\overbrace{Z}^{m\cdot g^{xy}}\cdot({\overbrace{Y^x}^{g^{xy}}})^{-1} = m$.
\end{itemize}

Not ind-cpa-secure for $G=\ZZ_p^*$.
Attack: $(Y,Z)\gets O(1,g)$.
Accept if $Z^{(p-1)/2}=1$, reject if $-1$.
$\Pr[\mathcal{A}^L]=\Pr[1\cdot g^{xy}\;\text{is square}]=3/4$.
$\Pr[\mathcal{A}^R]=\Pr[g^{xy+1}\;\text{is square}]=1/4$.

Is ind-cpa-secure under assumption that DDH (decision Diffie-Hellman) is hard.

Is not cca-secure.
But the Cramer-Shoup scheme is,
using two applications of ElGamal,
under the DDH assumption.

\end{document}
